\documentclass{resume}

\newcommand{\en}[1]{#1}
\newcommand{\zh}[1]{}

\zh{\usepackage{xeCJK}}
\zh{\setCJKmainfont{Source Han Serif CN}}
\zh{\setCJKsansfont{Source Han Sans CN}}
\zh{\setCJKmonofont{Source Han Sans CN}}

\begin{document}

\name{\en{Rustin Liu}\zh{刘东坡}}
\basicInfo{
      \email{rustin.liu@gmail.com} \textperiodcentered\
      \phone{+(86) 189-0839-4732} \textperiodcentered\
      \github[hi-rustin]{https://github.com/hi-rustin} \textperiodcentered\
      \homepage[Website]{https://hi-rustin.rs/}
}

\section{\faUsers\ \en{Work Experience}\zh{工作经历}}
\en{\datedsubsection{\textbf{\href{https://pingcap.com/en/}{PingCAP Inc.}}}{August 2020 -- Present}}
\zh{\datedsubsection{\textbf{\href{https://pingcap.com/zh/}{北京平凯星辰科技发展有限公司(PingCAP Inc.)}}}{2020/08 -- 至今}}
\en{\role{Database R\&D}{\href{https://github.com/pingcap/tiflow}{TiCDC} R\&D}}
\zh{\role{数据库研发}{\href{https://github.com/pingcap/tiflow}{TiCDC} 研发工程师}}
\begin{itemize}
      \item \en{Rewrites TiCDC core data synchronization process, modifying the original push mode to pull mode to optimize memory management and data synchronization process. Provides more flexible task-level memory control, greatly improving OOM issues and throughput.}
            \zh{重写 TiCDC 核心数据同步流程,将原来的推送模式修改为拉取模式优化了内存管理和数据同步流程。提供了更灵活的任务界别的内存控制,极大的改善了 OOM 问题和吞吐。}
      \item \en{Rewrote the TiCDC Sink module to remove a lot of useless and harmful abstractions, changed from synchronous mode to asynchronous mode, and optimized code performance to improve throughput by 30\%.}
            \zh{重写 TiCDC Sink 模块,删除了大量无用有害抽象,从同步模式修改为异步模式,优化代码性能,将吞吐提升 30\%。}
      \item \en{Responsible for supporting the Multi-Topic feature for TiCDC Kafka Sink, allowing it to send data messages to multiple Topics based on table information.}
            \zh{为 TiCDC Kafka Sink 支持了多 Topic 功能,让其可以根据表信息将数据消息发送至多个 Topics。}
      \item \en{Responsible for refactoring all of TiCDC's CLI code, removing a lot of global variables and dead code, standardizing how commands are added and written, and greatly improving maintainability.}
            \zh{负责重构了 TiCDC 的 CLI 所有代码,删除了大量全局变量和死代码,标准化命令添加和编写方式,极大提升了可维护性。}
      \item \en{Ported the project's integration tests to Docker so that they could run locally, greatly improving development efficiency. I also optimized the test scripts and reduced CI time consumption from 40 minutes to 20 minutes.}
            \zh{将该项目的集成测试移植到 Docker 中,使其能在本地运行,极大的提升了开发效率。同时优化了测试脚本,将 CI 时间消耗从 40 分钟降低至 20 分钟。}
\end{itemize}
\en{\role{Community Front-end and Back-end R\&D}{\href{https://github.com/ti-community-infra/tichi}{TiChi} R\&D}}
\zh{\role{社区前后端开发工程师}{\href{https://github.com/ti-community-infra/tichi}{TiChi} 研发}}
\begin{itemize}
      \item \en{Responsible for designing and building the entire project, building collaborative robots on GKE using Prow.}
            \zh{负责设计和构建整个项目,使用 Prow 在 GKE 上搭建协作机器人。}
      \item \en{Fixed bugs and added features for Prow, submitted 39 PRs.}
            \zh{为 Prow 修复 bug 和添加功能,提交了 39 个 PR。}
\end{itemize}

\en{\datedsubsection{\textbf{\href{https://www.morningstar.com/}{Morningstar, Inc.}}}{June 2019 -- July 2020}}
\zh{\datedsubsection{\textbf{\href{https://www.morningstar.com/}{晨星资讯(Morningstar, Inc. )}}}{2019/06 -- 2020/07}}
\en{\role{Front-end and Back-end R\&D}{Components}}
\zh{\role{前后端开发工程师}{组件开发}}
\begin{itemize}
      \item \en{Responsible for writing and maintaining some components, developed using springboot and Vue. Several components are already online, with good stability and performance.}
            \zh{负责独立组件的编写和维护,使用 springboot 和 Vue 开发。多个组件已经上线使用,有较好的稳定性和性能。}
      \item \en{Responsible for designing the tracking system design and implementation, using HeapIO to track user information, with a scalable interface to seamlessly switch to Google Analytics.}
            \zh{负责设计和实现前端 tracking 系统,使用 HeapIO 追踪用户信息,并预留可扩展接口,可无缝切换至 Google Analytics。}
      \item \en{Wrote automated Jenkins scripts for component and document releases, optimizing the release process from a manual 4-hour process to a fully automated 30-minute process.}
            \zh{为组件和文档发布编写了自动化 Jenkins 脚本,将发布流程从手动 4 小时优化到全自动 30 分钟。}
      \item \en{Optimized the company's front-end scaffolding to support TypeScript, added prettier to unify code formatting, and unify ESLint rules.}
            \zh{优化公司前端脚手架,使其支持 TypeScript,添加 prettier 统一代码格式,优化统一 ESLint 规则。}
\end{itemize}

\section{\faGraduationCap\ \en{Education}\zh{教育经历}}
\en{\datedsubsection{\textbf{Software Engineering Bachelor's Degree}}{Sep 2015 -- June 2019}}
\zh{\datedsubsection{\textbf{软件工程 本科}}{2015/09 -- 2019/06}}
\begin{itemize}
      \item \en{Chongqing University of Technology}
            \zh{重庆理工大学}
\end{itemize}

\section{\faGithubAlt\ \en{Open Source Projects}\zh{开源项目}}
\datedsubsection{\textbf{Cargo \& Rustup}}{\url{https://github.com/rust-lang}}
\en{\role{Active Maintainer}{\href{https://github.com/search?q=repo:rust-lang/cargo+repo:rust-lang/rustup++author:hi-rustin&type=commits}{260+ commits}}}
\zh{\role{活跃维护者}{\href{https://github.com/search?q=repo:rust-lang/cargo+repo:rust-lang/rustup++author:hi-rustin&type=commits}{260+ 个提交}}}
\begin{itemize}
      \item \en{Improved various error prompts for Rustup and Cargo commands to make errors to effective and clear.}
            \zh{改善了大量 Rustup 和 Cargo 命令的错误提示,让错误更有效更清晰。}
      \item \en{Added the --crate-type function and added related documentation for Cargo.}
            \zh{为 Cargo 添加了 --crate-type 功能和相关文档。}
      \item \en{Different modes of self-update were supported for Rustup to provide a better experience.}
            \zh{为 Rustup 支持了 self-update 的不同模式来提供更好的使用体验。}
      \item \en{Improved Rustup testing and was responsible for releasing Rustup.}
            \zh{改善 Rustup 测试,负责发布 Rustup。}
\end{itemize}

\datedsubsection{\textbf{Phlare}}{\url{https://github.com/grafana/phlare}}
\en{\role{Active Contributor}{\href{https://github.com/grafana/phlare/commits?author=hi-rustin}{28+ commits}}}
\zh{\role{活跃贡献者}{\href{https://github.com/grafana/phlare/commits?author=hi-rustin}{28+ 个提交}}}
\begin{itemize}
      \item \en{Refactored all command line flags and added flags descriptions and tests}
            \zh{重构了所有命令行参数,并添加了参数说明和测试}
      \item \en{Added a management page to show the status of each component}
            \zh{添加了管理页面展示各组件状态}
      \item \en{Refactored all APIs to make them easier to write and test}
            \zh{重构了所有 API 使其更容易编写和测试}
\end{itemize}

\datedsubsection{\textbf{ant-design}}{\url{https://github.com/ant-design/ant-design}}
\en{\role{Previous Maintainer}{\href{https://github.com/ant-design/ant-design/commits?author=hi-rustin}{24+ commits}}}
\zh{\role{前维护者}{\href{https://github.com/ant-design/ant-design/commits?author=hi-rustin}{24+ 个提交}}}
\begin{itemize}
      \item \en{Refactored Input component and added cleanup button}
            \zh{重构了 Input 组件,添加清理按钮}
      \item \en{Designed and added Skeleton component}
            \zh{设计并添加了 Skeleton 组件}
\end{itemize}

\section{\faCogs\ \en{Skills}\zh{技能}}
\begin{itemize}[parsep=0.25ex]
      \item \en{\textbf{Programming Languages}:
                  Experienced in Go/Rust/JavaScript,
                  \textbf{multilingual} (can adapt to any language)}
            \zh{\textbf{编程语言}:
                  熟悉 Go/Rust/JavaScript,
                  \textbf{泛语言}(编程不受特定语言限制)}

      \item \en{\textbf{Development Tools}:
                  familiar with Linux-based programming,
                  have experience with team tools like Git, GitHub, Slack, and Jira, etc.}
            \zh{\textbf{开发工具}:
                  熟悉 Linux,有 Git、GitHub、Slack 和 Jira 等团队协作工具的使用经验}

      \item \en{\textbf{Languages}:
                  \textbf{English} (Can read, write and speak fluently),
                  \textbf{Chinese} (Native speaker)}
            \zh{\textbf{语言}:
                  \textbf{英语} (可以流利的听说读写),
                  \textbf{中文} (母语)}
\end{itemize}

\end{document}
