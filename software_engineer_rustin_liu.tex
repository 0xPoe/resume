\documentclass{software_engineer_rustin_liu}

\newcommand{\en}[1]{#1}
\newcommand{\zh}[1]{}

\zh{\usepackage{xeCJK}}
\zh{\setCJKmainfont{Source Han Serif CN}}
\zh{\setCJKsansfont{Source Han Sans CN}}
\zh{\setCJKmonofont{Source Han Sans CN}}

\begin{document}

\name{\en{Rustin Liu}\zh{刘东坡}}
\basicInfo{
      \email{rustin.liu@gmail.com} \textperiodcentered\
      \phone{+(86) 189-0839-4732} \textperiodcentered\
      \github[hi-rustin]{https://github.com/hi-rustin} \textperiodcentered\
      \homepage[hi-rustin.rs]{https://hi-rustin.rs/}
}

\section{\faGraduationCap\ \en{Education}\zh{教育经历}}
\en{\datedsubsection{\textbf{Software Engineering Bachelor's Degree}}{Sep 2015 -- June 2019}}
\zh{\datedsubsection{\textbf{软件工程 本科}}{2015/09 -- 2019/06}}
\begin{itemize}
      \item \en{Chongqing University of Technology}
            \zh{重庆理工大学}
\end{itemize}

\section{\faCogs\ \en{Skills}\zh{技能}}
\begin{itemize}[parsep=0.25ex]
      \item \en{\textbf{Programming Languages}:
                  Experienced in Go/Rust/JavaScript,
                  \textbf{multilingual} (can adapt to any language)}
            \zh{\textbf{编程语言}:
                  熟悉 Go/Rust/JavaScript
                  \textbf{泛语言}(编程不受特定语言限制)}

      \item \en{\textbf{Development Tools}:
                  familiar with Linux-based programming,
                  have experience with tools like Git, GitHub, Docker, Jenkins, Github Actions and Jira, etc.}
            \zh{\textbf{开发工具}:
                  熟悉 Linux,有 Git、GitHub、Docker、Jenkins、Github Actions 和 Jira 等团队协作工具的使用经验}

      \item \en{\textbf{Languages}:
                  \textbf{English} (Can read, write and speak fluently),
                  \textbf{Chinese} (Native speaker)}
            \zh{\textbf{语言}:
                  \textbf{英语} (可以流利的听说读写),
                  \textbf{中文} (母语)}
\end{itemize}

\section{\faUsers\ \en{Work Experience}\zh{工作经历}}
\en{\datedsubsection{\textbf{\href{https://pingcap.com/en/}{PingCAP Inc. - Database - Database R\&D}}}{July 2021 -- Present}}
\zh{\datedsubsection{\textbf{\href{https://pingcap.com/zh/}{PingCAP Inc. - 数据库 - 数据库研发}}}{2021/07 -- 至今}}
\en{\textsl{Transferred from the community department to the data platform department and started to focus on the development of data synchronization tools {\href{https://github.com/pingcap/tiflow}{TiCDC}}.}}
\zh{\textsl{从社区部门转组到数据平台部门,开始专注于数据同步工具 {\href{https://github.com/pingcap/tiflow}{TiCDC} 的研发。}}}

\en{\textbf{Responsibilities:}}
\zh{\textbf{职责:}}
\begin{itemize}
      \item \en{Designed, implemented and maintained core modules, mainly focusing on Sink and Sorter modules.}
            \zh{设计、实现和维护核心模块,主要聚焦于 Sink 和 Sorter 模块。}
      \item \en{Provided day-to-day maintenance of the project: Tracked issues and resolved them. Reviewed code and was on-call to resolve customer issues.}
            \zh{项目日常的维护:Issues 追踪和解决、代码审阅和 on-call 解决客户问题。}
      \item \en{Optimized project code structure and CI/CD script. Tracked flaky test to improve engineering efficiency.}
            \zh{优化项目代码结构、追踪不稳定测试,优化 CI/CD 脚本,提升工程效率。}
\end{itemize}

\en{\textbf{Accomplishments:}}
\zh{\textbf{产出:}}
\begin{itemize}
      \item \en{Changed push mode to pull mode to optimize memory management and data synchronization process, greatly improving OOM issues and synchronization lags.}
            \zh{将推送模式修改为了拉取模式优化了内存管理和数据同步流程,极大的改善了 OOM 问题和同步延迟。}
      \item \en{Modified the Sink module from synchronous to asynchronous mode to optimize code performance and improve throughput by 30\% in incremental scenarios.}
            \zh{将 Sink 模块从同步模式修改为异步模式,优化代码性能,将增量场景下吞吐提升 30\%。}
      \item \en{ Added multi-topic and OAuth authorization support for Kafka Sink, greatly reducing the operation and maintenance costs of multi-topic tasks.}
            \zh{为 Kafka Sink 支持了多 Topic 和 OAuth 授权功能,极大的降低了多 Topic 的任务运维成本。}
      \item \en{Added the ability to optimize the Lossy DDL from full table scan to not synchronizing any data, greatly reducing the impact of DDL on the system.}
            \zh{将有损 DDL 从全表扫描优化为不同步任何数据,极大的降低了有损 DDL 对系统的影响。}
      \item \en{Refactored all of TiCDC’s CLI code to standardize the way commands are added and written, which greatly improved maintainability.}
            \zh{重构了 TiCDC 的 CLI 所有代码,标准化了命令添加和编写方式,极大提升了可维护性。}
\end{itemize}
\en{\datedsubsection{\textbf{\href{https://pingcap.com/en/}{PingCAP Inc. - Database - Front-end/Back-end R\&D}}}{Aug 2020 -- July 2021}}
\zh{\datedsubsection{\textbf{\href{https://pingcap.com/zh/}{PingCAP Inc. - 数据库 - 前后端开发工程师}}}{2020/08 -- 2021/07}}
\en{\textsl{Joined the community department to develop {\href{https://github.com/ti-community-infra/tichi}{TiChi}} to support community collaboration based on the Kubernetes Community{\href{https://github.com/kubernetes/test-infra/tree/master/prow}{ Prow}} project.}}
\zh{\textsl{加入了社区部门,以 Kubernetes 社区{\href{https://github.com/kubernetes/test-infra/tree/master/prow}{ Prow}} 项目为基础开发{\href{https://github.com/ti-community-infra/tichi}{ TiChi}} 来支撑社区协作。}}

\en{\textbf{Responsibilities:}}
\zh{\textbf{职责:}}
\begin{itemize}
      \item \en{Designed and built the entire project. Built collaborative robots on GKE using Prow.}
            \zh{从零开始设计和构建整个项目,使用 Prow 在 GKE 上搭建协作机器人。}
      \item \en{Sorted out and optimized the community collaboration process to provide a better collaboration experience for open source contributors.}
            \zh{负责社区协作流程的梳理和优化,为开源贡献者提供更好的协作体验。}
\end{itemize}

\en{\textbf{Accomplishments:}}
\zh{\textbf{产出:}}
\begin{itemize}
      \item \en{Completed TiChi, which has become the company’s CI/CD infrastructure, on time.}
            \zh{按时完成了项目的设计和开发,TiChi 已经成为了公司 CI/CD 的基础设施。}
      \item \en{Fixed bugs and added features for Prow, submitted 39 PRs.}
            \zh{为 Prow 修复 bug 和添加功能,提交了 39 个 PR。}
      \item \en{Added multiple language documents for collaboration process and TiChi.}
            \zh{为 TiChi 编写了良好的文档,明确了协作流程,为开源贡献者提供了更好的协作体验。}
\end{itemize}

\en{\datedsubsection{\textbf{\href{https://www.morningstar.com/}{Morningstar, Inc. - Financial Services - Front-end/Back-end R\&D}}}{June 2019 -- July 2020}}
\zh{\datedsubsection{\textbf{\href{https://www.morningstar.com/}{晨星资讯(Morningstar, Inc.)- 金融服务 - 前后端开发工程师}}}{2019/06 -- 2020/07}}
\en{\textsl{First job after graduation, responsible for writing and maintaining data presentation components, left after the company decided to start closing its China R\&D center.}}
\zh{\textsl{毕业后第一份工作,负责编写和维护数据展示组件,在公司决定开始关闭中国研发中心后离职。}}

\en{\textbf{Responsibilities:}}
\zh{\textbf{职责:}}
\begin{itemize}
      \item \en{Developed front and back-end of the data presentation component in Java and Vue.}
            \zh{负责数据展示组件的前后端开发,使用 Java 和 Vue 开发。}
      \item \en{Optimized the release scripts for components and documents to improve engineering efficiency.}
            \zh{负责组件和文档发布脚本的优化,提升工程效率。}
      \item \en{Maintained and developed front-end infrastructure, driving front-end engineering.}
            \zh{负责前端基础设施维护和开发,尝试和推动前端工程化。}
\end{itemize}

\en{\textbf{Accomplishments:}}
\zh{\textbf{产出:}}
\begin{itemize}
      \item \en{Independently developed components that are already online, with good stability and performance.}
            \zh{多个独立开发的组件已经上线使用,有较好的稳定性和性能。}
      \item \en{Implemented a front-end tracking system that uses HeapIO to track user information, with an extensible interface to seamlessly switch to Google Analytics.}
            \zh{实现了前端的 tracking 系统,使用 HeapIO 追踪用户信息,并预留可扩展接口,可无缝切换至 Google Analytics。}
      \item \en{Wrote automated Jenkins scripts for component and document releases, optimizing the release process from a manual 4-hour process to a fully automated 30-minute process.}
            \zh{为组件和文档发布编写了自动化 Jenkins 脚本,将发布流程从手动 4 小时优化到全自动 30 分钟。}
\end{itemize}

\section{\faGithubAlt\ \en{Open Source Projects}\zh{开源项目}}
\en{\datedsubsection{\textbf{Cargo \& Rust - Active Contributor}}{{\href{https://github.com/search?q=repo:rust-lang/cargo+repo:rust-lang/rust+author:hi-rustin&type=commits}{219+ commits}}}}
\zh{\datedsubsection{\textbf{Cargo \& Rust - 活跃贡献者}}{{\href{https://github.com/search?q=repo:rust-lang/cargo+repo:rust-lang/rust+author:hi-rustin&type=commits}{219+ commits}}}}
\en{\textsl{The Rust package manager. Cargo downloads your Rust package’s dependencies, compiles your packages, makes distributable packages.}}
\zh{\textsl{Rust 包管理器,Cargo 下载 Rust 包的依赖,编译包,制作可分发的包。}}

\begin{itemize}
      \item \en{Improved various error prompts for Cargo commands and the Rust compiler to make errors to effective and clear.}
            \zh{改善了大量 Cargo 命令和 Rust 编译器的错误提示,让错误更有效更清晰。}
      \item \en{Added the --crate-type function and added related documentation for Cargo.}
            \zh{为 Cargo 添加了 --crate-type 功能和相关文档。}
      \item \en{Added support for automatic inheritance of the [workspace.package] fields for the cargo new/init commands.}
            \zh{为 cargo new 和 cargo init 命令支持了 [workspace.package] 字段自动继承。}
      \item \en{Added new syntax for Cargo build scripts to provide a foundation for future extensions to build scripts.}
            \zh{为 Cargo 构建脚本添加新的语法,为未来扩展构建脚本提供基础。}
      \item \en{Added OR\_PATTERNS\_BACK\_COMPAT lint to help users migrate to the new or\-patterns syntax.}
            \zh{添加了 OR\_PATTERNS\_BACK\_COMPAT lint 来帮助用户迁移至新的 or\-patterns 语法。}
      \item \en{Made authors field optional in Cargo, crates.io and rustdoc.}
            \zh{使 authors 字段在 Cargo、crates.io 和 rustdoc 中可选。}
\end{itemize}

\en{\datedsubsection{\textbf{Rustup - Active Maintainer}}{{\href{https://github.com/rust-lang/rustup/commits?author=hi-rustin}{110+ commits}}}}
\zh{\datedsubsection{\textbf{Rustup - 活跃维护者}}{{\href{https://github.com/rust-lang/rustup/commits?author=hi-rustin}{110+ commits}}}}
\en{\textsl{Rustup is an installer for Rust. Rustup installs The Rust Programming Language from the official release channels, enabling you to easily switch between stable, beta, and nightly compilers and keep them updated.}}
\zh{\textsl{Rustup 是 Rust 的安装器。Rustup 从官方发布渠道安装 Rust 编程语言,使用户可以轻松地在稳定版、测试版和夜版编译器之间切换并保持更新。}}

\begin{itemize}
      \item \en{Different modes of self-update were supported for Rustup to provide a better experience.}
            \zh{为 Rustup 支持了 self-update 的不同模式来提供更好的使用体验。}
      \item \en{Improved Rustup testing and was responsible for releasing Rustup.}
            \zh{改善 Rustup 测试,负责发布 Rustup。}
      \item \en{Added UI tests for Rustup and successfully upgraded the clap dependency.}
            \zh{为 Rustup 添加了 UI 测试,顺利升级了 clap 依赖。}
      \item \en{Responsible for the daily maintenance of the project, including code review, security vulnerability fixing, etc.}
            \zh{负责项目日常维护,包括代码审查、依赖升级、安全漏洞修复等。}
\end{itemize}

\en{\datedsubsection{\textbf{ant-design - Previous Maintainer}}{{\href{https://github.com/ant-design/ant-design/commits?author=hi-rustin}{24+ commits}}}}
\zh{\datedsubsection{\textbf{ant-design - 前维护者}}{{\href{https://github.com/ant-design/ant-design/commits?author=hi-rustin}{24+ commits}}}}
\en{\textsl{An enterprise-class UI design language and React UI library.}}
\zh{\textsl{企业级 UI 设计语言和 React UI 库。}}

\begin{itemize}
      \item \en{Refactored Input component and added cleanup button.}
            \zh{重构了 Input 组件,添加清理按钮。}
      \item \en{Designed and added Skeleton component.}
            \zh{设计并添加了 Skeleton 组件。}
      \item \en{Wrote contribution guidelines to guide others to participate in community contributions.}
            \zh{编写了贡献指南,引导其他人参与社区贡献。}
\end{itemize}

\end{document}
