\documentclass{software_engineer_rustin_liu}

\newcommand{\en}[1]{#1}
\newcommand{\zh}[1]{}

\zh{\usepackage{xeCJK}}
\zh{\setCJKmainfont{Heiti SC}}
\zh{\setCJKsansfont{Heiti SC}}
\zh{\setCJKmonofont{Menlo}}

\begin{document}

\setlength{\baselineskip}{1.2\baselineskip}

\name{\en{Rustin Liu}\zh{刘东坡}}
\basicInfo{
      \email{rustin.liu@gmail.com} \textperiodcentered\
      \phone{+(86) 189-0839-4732} \textperiodcentered\
      \github[Rustin170506]{https://github.com/Rustin170506} \textperiodcentered\
      \homepage[Rustin170506.rs]{https://rustin.me/}
}

\section{\faGraduationCap\ \en{Education}\zh{教育经历}}
\en{\datedsubsection{\textbf{Software Engineering Bachelor's Degree}}{Sep 2015 -- June 2019}}
\zh{\datedsubsection{\textbf{软件工程 本科}}{2015/09 -- 2019/06}}
\begin{itemize}
      \item \en{Chongqing University of Technology}
            \zh{重庆理工大学}
\end{itemize}

\section{\faCogs\ \en{Skills}\zh{技能}}
\begin{itemize}[parsep=0.25ex]
      \item \en{\textbf{Programming Languages}:
                  Experienced in Go/Rust/JavaScript,
                  \textbf{multilingual} (can adapt to any language)}
            \zh{\textbf{编程语言}:
                  熟悉 Go/Rust/JavaScript
                  \textbf{泛语言}(编程不受特定语言限制)}

      \item \en{\textbf{Development Tools}:
                  familiar with Linux-based programming,
                  have experience with tools like Git, GitHub, Docker, Jenkins, Github Actions and Jira, etc.}
            \zh{\textbf{开发工具}:
                  熟悉 Linux,有 Git、GitHub、Docker、Jenkins、Github Actions 和 Jira 等团队协作工具的使用经验}

      \item \en{\textbf{Languages}:
                  \textbf{English} (Can read, write and speak fluently),
                  \textbf{Chinese} (Native speaker)}
            \zh{\textbf{语言}:
                  \textbf{英语} (可以流利的听说读写),
                  \textbf{中文} (母语)}
\end{itemize}

\section{\faUsers\ \en{Work Experience}\zh{工作经历}}

\en{\datedsubsection{\textbf{\href{https://pingcap.com/en/}{PingCAP Inc. - Database - Database Kernel R\&D(Golang/Rust)}}}{August 2023 -- Present}}
\zh{\datedsubsection{\textbf{\href{https://pingcap.com/zh/}{PingCAP Inc. - 数据库 - 数据库内核研发(Golang/Rust)}}}{2023/08 -- 至今}}
\en{\textsl{Transferred from the data platform department to the kernel computing department and started to focus on the development of the optimizer of the distributed database {\href{https://github.com/pingcap/tidb}{TiDB}}.}}
\zh{\textsl{从数据平台部门转组到计算内核部门,开始专注于分布式数据库 {\href{https://github.com/pingcap/tidb}{TiDB} 优化器的研发。}}}

\en{\textbf{Responsibilities:}}
\zh{\textbf{职责:}}
\begin{itemize}
      \item \en{Designed, implemented and maintained the optimizer of the distributed database TiDB.}
      \zh{设计、实现和维护分布式数据库 TiDB 的优化器。}
\end{itemize}

\en{\textbf{Accomplishments:}}
\zh{\textbf{产出:}}
\begin{itemize}
      \item \en{Optimized statistics collection by implementing an in-memory priority queue, eliminating repeated queue rebuilding overhead, reducing CPU usage by one core and memory consumption by 99\%.}
      \zh{通过实现内存中的优先级队列优化统计信息收集,消除重复队列重建开销,减少一个 CPU 核心使用率并降低 99\% 内存消耗。}
      \item \en{Designed and implemented the predicate columns feature to optimize statistics collection by only analyzing columns used in predicates, reducing auto analyze time by 50\% in OLTP workloads and nearly 99\% in partitioned wide table scenarios.}
      \zh{设计和实现了 predicate columns 功能,通过只分析谓词中使用的列来优化统计信息收集,在 OLTP 工作负载下将自动分析时间减少了 50\%,在分区宽表场景下提升近 99\%。}
      \item \en{Designed and implemented the priority queue for speed up the auto analyze process, significantly reducing the starvations of the auto analyze process.}
      \zh{设计并实现了优先级队列,加速了自动分析过程,极大的减少了自动分析过程的饥饿现象。}
      \item \en{Added support for the LOCK STATS TABLE feature to allow users to lock statistics updates and generate stable execution plans.}
      \zh{添加 LOCK STATS TABLE 功能,方便用户锁定统计信息的更新,生成稳定的执行计划。}
\end{itemize}

\en{\datedsubsection{\textbf{\href{https://pingcap.com/en/}{PingCAP Inc. - Database - Database Tools R\&D(Golang/Rust)}}}{July 2021 -- July 2023}}
\zh{\datedsubsection{\textbf{\href{https://pingcap.com/zh/}{PingCAP Inc. - 数据库 - 数据库工具研发(Golang/Rust)}}}{2021/07 -- 2023/07}}
\en{\textsl{Transferred from the community department to the data platform department and started to focus on the development of data synchronization tools {\href{https://github.com/pingcap/tiflow}{TiCDC}}.}}
\zh{\textsl{从社区部门转组到数据平台部门,开始专注于数据同步工具 {\href{https://github.com/pingcap/tiflow}{TiCDC} 的研发。}}}

\en{\textbf{Responsibilities:}}
\zh{\textbf{职责:}}
\begin{itemize}
      \item \en{Designed, implemented, and maintained core modules, focusing on Sink and Sorter, and managed daily project maintenance including issue tracking, code reviews, and on-call customer support.}
            \zh{设计、实现和维护核心模块,专注于 Sink 和 Sorter,管理项目日常维护,包括 issue 追踪、代码审阅和 on-call 客户支持。}
\end{itemize}

\en{\textbf{Accomplishments:}}
\zh{\textbf{产出:}}
\begin{itemize}
      \item \en{Changed push mode to pull mode to optimize memory management and data synchronization process, greatly improving OOM issues and synchronization lags.}
            \zh{将推送模式修改为了拉取模式优化了内存管理和数据同步流程,极大的改善了 OOM 问题和同步延迟。}
      \item \en{Modified the Sink module from synchronous to asynchronous mode to optimize code performance and improve throughput by 30\% in incremental scenarios.}
            \zh{将 Sink 模块从同步模式修改为异步模式,优化代码性能,将增量场景下吞吐提升 30\%。}
      \item \en{ Added multi-topic and OAuth authorization support for Kafka Sink, greatly reducing the operation and maintenance costs of multi-topic tasks.}
            \zh{为 Kafka Sink 支持了多 Topic 和 OAuth 授权功能,极大的降低了多 Topic 的任务运维成本。}
      \item \en{Added the ability to optimize the Lossy DDL from full table scan to not synchronizing any data, greatly reducing the impact of DDL on the system.}
            \zh{将有损 DDL 从全表扫描优化为不同步任何数据,极大的降低了有损 DDL 对系统的影响。}
      \item \en{Refactored all of TiCDC's CLI code to standardize the way commands are added and written, which greatly improved maintainability.}
            \zh{重构了 TiCDC 的 CLI 所有代码,标准化了命令添加和编写方式,极大提升了可维护性。}
\end{itemize}

\en{\datedsubsection{\textbf{\href{https://pingcap.com/en/}{PingCAP Inc. - Database - Full-Stack Engineer(Golang/JS)}}}{Aug 2020 -- July 2021}}
\zh{\datedsubsection{\textbf{\href{https://pingcap.com/zh/}{PingCAP Inc. - 数据库 - 前后端开发工程师(Golang/JS)}}}{2020/08 -- 2021/07}}
\en{\textsl{Joined the community department to develop {\href{https://github.com/ti-community-infra/tichi}{TiChi}} to support community collaboration based on the Kubernetes Community{\href{https://github.com/kubernetes/test-infra/tree/master/prow}{ Prow}} project.}}
\zh{\textsl{加入了社区部门,以 Kubernetes 社区{\href{https://github.com/kubernetes/test-infra/tree/master/prow}{ Prow}} 项目为基础开发{\href{https://github.com/ti-community-infra/tichi}{ TiChi}} 来支撑社区协作。}}

\en{\textbf{Responsibilities:}}
\zh{\textbf{职责:}}
\begin{itemize}
      \item \en{Designed and implemented a project with collaborative robots on GKE using Prow, optimizing community collaboration processes for open source contributors.}
            \zh{使用 Prow 在 GKE 上设计和实现一个协作机器人项目,优化开源贡献者的社区协作流程。}
\end{itemize}

\en{\textbf{Accomplishments:}}
\zh{\textbf{产出:}}
\begin{itemize}
      \item \en{Completed TiChi, which has become the company's CI/CD infrastructure, on time.}
            \zh{按时完成了项目的设计和开发,TiChi 已经成为了公司 CI/CD 的基础设施。}
      \item \en{Fixed bugs and added features for Prow, submitted 39 PRs.}
            \zh{为 Prow 修复 bug 和添加功能,提交了 39 个 PR。}
\end{itemize}

\en{\datedsubsection{\textbf{\href{https://www.morningstar.com/}{Morningstar, Inc. - Financial Services - Full-Stack Engineer(Java/JS)}}}{June 2019 -- July 2020}}
\zh{\datedsubsection{\textbf{\href{https://www.morningstar.com/}{晨星资讯(Morningstar, Inc.)- 金融服务 - 前后端开发工程师(Java/JS)}}}{2019/06 -- 2020/07}}
\en{\textsl{First job after graduation, responsible for writing and maintaining data presentation components, left after the company decided to start closing its China R\&D center.}}
\zh{\textsl{毕业后第一份工作,负责编写和维护数据展示组件,在公司决定开始关闭中国研发中心后离职。}}

\en{\textbf{Responsibilities:}}
\zh{\textbf{职责:}}
\begin{itemize}
      \item \en{Developed and optimized front and back-end components in Java and Vue, enhancing engineering efficiency and front-end infrastructure.}
            \zh{使用 Java 和 Vue 开发和优化前后端组件,提升工程效率和前端基础设施。}
\end{itemize}

\en{\textbf{Accomplishments:}}
\zh{\textbf{产出:}}
\begin{itemize}
      \item \en{Independently developed components that are already online, with good stability and performance.}
            \zh{多个独立开发的组件已经上线使用,有较好的稳定性和性能。}
      \item \en{Implemented a front-end tracking system that uses HeapIO to track user information, with an extensible interface to seamlessly switch to Google Analytics.}
            \zh{实现了前端的 tracking 系统,使用 HeapIO 追踪用户信息,并预留可扩展接口,可无缝切换至 Google Analytics。}
      \item \en{Wrote automated Jenkins scripts for component and document releases, optimizing the release process from a manual 4-hour process to a fully automated 30-minute process.}
            \zh{为组件和文档发布编写了自动化 Jenkins 脚本,将发布流程从手动 4 小时优化到全自动 30 分钟。}
\end{itemize}

\section{\faGithubAlt\ \en{Open Source Projects}\zh{开源项目}}
\en{\datedsubsection{\textbf{Cargo(Rust) - Maintainer}}{{\href{https://github.com/search?q=repo:rust-lang/cargo+repo:rust-lang/rust+author:Rustin170506&type=commits}{260+ commits}}}}
\zh{\datedsubsection{\textbf{Cargo(Rust) - 维护者}}{{\href{https://github.com/search?q=repo:rust-lang/cargo+repo:rust-lang/rust+author:Rustin170506&type=commits}{260+ commits}}}}
\en{\textsl{The Rust package manager. Cargo downloads your Rust package's dependencies, compiles your packages, makes distributable packages.}}
\zh{\textsl{Rust 包管理器,Cargo 下载 Rust 包的依赖,编译包,制作可分发的包。}}

\begin{itemize}
      \item \en{Implemented the `cargo-info` subcommand to display detailed information about a Rust package.}
            \zh{为 Cargo 实现了 `cargo-info` 子命令来展示包的详细信息。}
      \item \en{Improved various error prompts for Cargo commands and the Rust compiler to make errors to effective and clear.}
            \zh{改善了大量 Cargo 命令和 Rust 编译器的错误提示,让错误更有效更清晰。}
      \item \en{Added the --crate-type function and added related documentation for Cargo.}
            \zh{为 Cargo 添加了 --crate-type 功能和相关文档。}
      \item \en{Added support for automatic inheritance of the [workspace.package] fields for the cargo new/init commands.}
            \zh{为 cargo new 和 cargo init 命令支持了 [workspace.package] 字段自动继承。}
      \item \en{Added new syntax for Cargo build scripts to provide a foundation for future extensions to build scripts.}
            \zh{为 Cargo 构建脚本添加新的语法,为未来扩展构建脚本提供基础。}
      \item \en{Added `target.'cfg(..)'.linker` configuration option to allow users to specify the linker via `cfg(..)' conditions.}
            \zh{添加了 `target.'cfg(..)'.linker` 配置项,允许用户通过 `cfg(..)' 条件来指定链接器。}
      \item \en{Added OR\_PATTERNS\_BACK\_COMPAT lint to help users migrate to the new or\-patterns syntax.}
            \zh{添加了 OR\_PATTERNS\_BACK\_COMPAT lint 来帮助用户迁移至新的 or\-patterns 语法。}
      \item \en{Made authors field optional in Cargo and rustdoc.}
            \zh{使 authors 字段在 Cargo 和 rustdoc 中可选。}
\end{itemize}


\en{\datedsubsection{\textbf{crates.io(Rust/JS) - Maintainer}}{{\href{https://github.com/rust-lang/crates.io/commits?author=Rustin170506}{131+ commits}}}}
\zh{\datedsubsection{\textbf{crates.io(Rust/JS) - 维护者}}{{\href{https://github.com/rust-lang/crates.io/commits?author=Rustin170506}{131+ commits}}}}
\en{\textsl{The Rust package registry.}}
\zh{\textsl{Rust 包管理中心。}}

\begin{itemize}
      \item \en{Helped implement RFC 3052 to allow crates.io to accept packages without author information.}
            \zh{帮助实现 RFC 3052,让 crates.io 开始接受无作者信息的包。}
      \item \en{Added support for renaming packages with underscore prefixes to be published on crates.io.}
            \zh{支持下划线前缀的重命名包在 crates.io 的发布。}
      \item \en{Aligned crates.io and Cargo on the naming rules for features and crates.}
            \zh{对齐了 crates.io 和 Cargo 对 feature 和 crate 的名称校验规则。}
      \item \en{Implemented token expiration notifications and one-click token copying to improve user experience.}
            \zh{实现了 token 过期提醒和一键复制功能以提升用户体验。}
\end{itemize}


\en{\datedsubsection{\textbf{tokio-console-web(TypeScript/Rust) - Author}}{{\href{https://github.com/Rustin170506/tokio-console-web}{283+ commits}}}}
\zh{\datedsubsection{\textbf{tokio-console-web(TypeScript/Rust) - 作者}}{{\href{https://github.com/Rustin170506/tokio-console-web}{283+ commits}}}}
\en{\textsl{tokio-console-web is a web-based console for debuging and monitoring the tokio runtime.}}
\zh{\textsl{tokio-console-web 是一个基于 web 的控制台,用于调试和监控 tokio runtime。}}

\begin{itemize}
      \item \en{Displays all tasks in the tokio runtime and their status.}
            \zh{显示 tokio runtime 中的所有任务及其状态。}
      \item \en{Displays the details of each task, including a visual histogram of the polling(busy) times of the task.}
            \zh{显示每个任务的详细信息,包括任务的轮询(忙碌)时间的柱状图。}
      \item \en{Displays the resources, such as synchronization primitives, I/O resources, etc., used by each task.}
            \zh{显示每个任务使用的资源,如同步原语、I/O 资源等。}
\end{itemize}


\en{\datedsubsection{\textbf{Rustup(Rust) - Previous Maintainer}}{{\href{https://github.com/rust-lang/rustup/commits?author=Rustin170506}{110+ commits}}}}
\zh{\datedsubsection{\textbf{Rustup(Rust) - 前维护者}}{{\href{https://github.com/rust-lang/rustup/commits?author=Rustin170506}{110+ commits}}}}

\en{\datedsubsection{\textbf{Ant Design(TypeScript) - Previous Maintainer}}{{\href{https://github.com/ant-design/ant-design/commits?author=Rustin170506}{20+ commits}}}}
\zh{\datedsubsection{\textbf{Ant Design(TypeScript) - 前维护者}}{{\href{https://github.com/ant-design/ant-design/commits?author=Rustin170506}{20+ commits}}}}

\end{document}
